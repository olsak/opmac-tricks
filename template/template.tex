\chyph  % use (pdf)csplain
\input opmac

\def\code#1{\def\tmpb{#1}\edef\tmpb{\expandafter\stripm\meaning\tmpb\relax}%
   \expandafter\replacestrings\expandafter{\string\\}{\bslash}%
   \expandafter\replacestrings\bslash{}%
   \codeP{\leavevmode\hbox{\tt\tmpb}}}
\def\codeP#1{#1}
\def\stripm#1->#2\relax{#2}
\addprotect\code \addprotect\\ \addprotect\{ \addprotect\} 
\addprotect\^ \addprotect\_ \addprotect\~
\setcnvcodesA
\addto\cnvhook{\let\code=\codeP}

\ifx\pdfunidef\undefined\else \def\cnvhook #1#2{#2\precode \pdfunidef\tmp\tmp}\fi
\def\precode{\def\codeP##1{}\let\bslash=\Bslash
   \edef\tmp{\expandafter}\expandafter\precodeA\tmp\code{}}
\def\precodeA#1\code#2{\addto\tmp{#1}%
   \ifx\end#2\end \else
      \codeA{#2}%
      \expandafter\addto\expandafter\tmp\expandafter{\tmpb}%
      \expandafter\precodeA \fi
}
\let\codeA=\code

\input pdfuni

%%%%%%%%%%%%%%%%%%%%%%%%%%%%%%%%%%%%%%%%%%%%%%%%%%%%%%%%%%%%%%%%%%%%%%%%%%%%%%%

\typosize[11/13]


\def\<{<}
\let\normalangle=\<
\catcode`\<=13
\def<#1>{\hbox{$\langle$\it#1\/$\rangle$}}

\newtoks\gtthook  \newtoks\ltthook
\def\tthook{\the\gtthook \the\ltthook \global\ltthook{}}
\gtthook{
	\adef{<}{\normalangle}
    \typosize[9/11]
    \advance\hsize by 1in
    \parindent=0pt
}

\hyperlinks{\Blue}{\Green}
\def\tocborder{1 .8 0} 
\let\pgborder\tocborder
\let\citeborder\tocborder
\let\refborder\tocborder
\let\urlborder\tocborder

\insertoutline{$TOC$} \outlines{0}

\def\mac#1{{\tt\char`\\#1}}

\def\ukazka#1{
	\bigskip
	{
		\typoscale[\magstep5/\magstep5]
		\centerline{\noindent#1}
	}
	\bigskip
}

\def\AmSTeX{AmS\TeX{}}

\def\tg{\mathop{\rm tg}\nolimits}

\def\mnotehook{\kern-2.3em\localcolor\LightGrey\typosize[9/11]}
\mnotesize = 1in
\mnoteskip=20pt

\def\trick[#1]#2#3#4{
    \label[#1]
    \bracedparam\secc{#2}% modified by P.O., 2017 --- \secc #2\par
    \mnote{\Grey{\bf #1}\LightGrey\break {\em #3}\break #4}
    
}
\def\bracedparam#1{\csname\string#1:M\endcsname} % added by P.O., 2017

%%%%%%%%%%%%%%%%%%%%%%%%%%%%%%%%%%%%%%%%%%%%%%%%%%%%%%%%%%%%%%%%%%%%%%%%%%%%%%%


\def\crtop{\cr \noalign{\hrule height.6pt \kern2.5pt}}
\def\crbot{\cr \noalign{\kern2.5pt\hrule height.6pt}}
\def\crmid{\cr \noalign{\kern1pt\hrule\kern1pt}}
\def\tablinefil{\leaders\hrule height.2pt\hfil\vrule height1.7pt depth1.5pt width0pt }

\def\tableA#1#2{\offinterlineskip \def\tmpa{}\tabdata={\kern-.5em}\scantabdata#1\relax
   \halign\expandafter{\the\tabdata\kern-.5em\tabstrutA\cr#2\crcr}\egroup}


\newcount\tabline  \tabline=1
\def\crx{\crcr \ifodd\tabline \colortabline \fi
         \global\advance\tabline by1 }
\def\colortabline{\noalign{\localcolor\LightGrey
   \hrule height\ht\strutbox depth\dp\strutbox \kern-\ht\strutbox 
   \kern-\dp\strutbox}}
\def\tabiteml{\quad}\def\tabitemr{\quad}

\newdimen\vvalX \newdimen\vvalY 
\newdimen\newHt \newdimen\newDp \newdimen\newLt \newdimen\newRt
\let\oripdfsetmatrix=\pdfsetmatrix

\def\multiplyMxV #1 #2 #3 #4 {% matrix * (vvalX, vvalY)
   \tmpdim = #1\vvalX \advance\tmpdim by #3\vvalY
   \vvalY  = #4\vvalY \advance\vvalY  by #2\vvalX
   \vvalX = \tmpdim
}
\def\multiplyMxM #1 #2 #3 #4 {% currmatrix := currmatrix * matrix
   \vvalX=#1pt \vvalY=#2pt \expandafter\multiplyMxV \currmatrix
   \edef\tmpb{\expandafter\ignorept\the\vvalX\space \expandafter\ignorept\the\vvalY}%
   \vvalX=#3pt \vvalY=#4pt \expandafter\multiplyMxV \currmatrix
   \edef\currmatrix{\tmpb\space 
      \expandafter\ignorept\the\vvalX\space \expandafter\ignorept\the\vvalY\space}%
}
\def\transformbox#1#2{\hbox{\setbox0=\hbox{#2}\preptransform{#1}%
   \kern-\newLt \vrule height\newHt depth\newDp width0pt
   \ht0=0pt \dp0=0pt \pdfsave#1\rlap{\box0}\pdfrestore \kern\newRt}%
}
\def\preptransform #1{\def\currmatrix{1 0 0 1 }%
   \def\pdfsetmatrix##1{\edef\tmpb{##1 }\expandafter\multiplyMxM \tmpb\unskip}#1%
   \setnewHtDp 0pt  \ht0  \setnewHtDp 0pt  -\dp0
   \setnewHtDp \wd0 \ht0  \setnewHtDp \wd0 -\dp0
   \let\pdfsetmatrix=\oripdfsetmatrix
}
\def\setnewHtDp #1 #2 {%
   \vvalX=#1\relax \vvalY=#2\relax \expandafter\multiplyMxV \currmatrix
   \ifdim\vvalX\<\newLt \newLt=\vvalX \fi \ifdim\vvalX>\newRt \newRt=\vvalX \fi  
   \ifdim\vvalY>\newHt \newHt=\vvalY \fi \ifdim-\vvalY\<\newDp \newDp=-\vvalY \fi
}

\let\hyphentt=\tentt  \tmpdim=\fontdimen6\hyphentt \divide\tmpdim by2

\def\hyphenprocess#1{\def\tmp{#1}\let\listwparts=\undefined 
   \setbox0=\vbox\bgroup\hyphenpenalty=-10000 \hsize=0pt \hfuzz=\maxdimen
      \def~{\nobreak\hskip.5em\relax}\tt\noindent\hskip0pt\relax #1\par
      \hyphenprocessA
   \expandafter\let\expandafter\listwparts\expandafter\empty
      \expandafter\hyphenprocessB\listwparts
}
\def\hyphenprocessA{\setbox2=\lastbox 
   \ifvoid2 \egroup \else \unskip \unpenalty 
      \setbox2=\hbox{\unhbox2}%
      \tmpnum=\wd2 \advance\tmpnum by100 \divide\tmpnum by\tmpdim
      \ifx\listwparts\undefined \xdef\listwparts{,}%
      \else \advance\tmpnum by-1 \xdef\listwparts{\the\tmpnum,\listwparts}\fi
              \expandafter\hyphenprocessA \fi
}
\def\hyphenprocessB#1,{\if^#1^\expandafter\hyphenprocessC
   \else \tmpnum=#1 \expandafter\hyphenprocessD\tmp\end
   \fi
}
\def\hyphenprocessC{\expandafter\addto\expandafter\listwparts\expandafter{\tmp}}   
\def\hyphenprocessD#1#2\end{\addto\listwparts{#1}%
   \advance\tmpnum by-1
   \ifnum\tmpnum>0 \def\next{\hyphenprocessD#2\end}%
   \else
   \def\tmp{#2}\def\next{\addto\listwparts{\-}\expandafter\hyphenprocessB}\fi
   \next
}


{

\mathchardef\widetildemax="0367


\def\widetildeto #1{\bgroup\tmpdim=#1\setbox0=\hbox{$\widetildemax$}%
   \tmpdim=16\tmpdim \tmpnum=\tmpdim \tmpdim=\wd0 \divide\tmpdim by16
   \divide\tmpnum by\tmpdim
   \hbox to#1{\pdfsave\rlap{\pdfscale{\the\tmpnum}{\ifnum\tmpnum&lt;588 1\else\the\tmpnum\fi}%
                            \pdfscale{.00390625}{\ifnum\tmpnum&lt;588 1\else.0017\fi}%
                            \vbox to0pt{\hbox{$\widetildemax$}\vss}}\pdfrestore\hss}%
   \egroup
}
\def\overtilde#1{\setbox1=\hbox{$#1$}%
  \vbox{\offinterlineskip \halign{\hfil##\hfil\cr
        \widetildeto{\wd1}\cr\noalign{\kern.5ex\kern.02\wd1}\box1\cr}}%
}

}

\def\coltextstrut{height2ex depth.6ex}
\def\coltext#1#2#3{{\localcolor\let\Tcolor=#1\let\Bcolor=#2\relax
   \setbox1=\hbox{-}%
   \setbox1=\hbox{{\Bcolor\vrule\coltextstrut width\wd1}\llap{\Tcolor -}}%
   \def\-{\discretionary{\copy1}{}{}}%
   \def\uline##1{\bgroup\Bcolor\leaders \vrule\coltextstrut\hskip##1\egroup}%
   \def\uspace{\fontdimen2\font plus\fontdimen3\font minus\fontdimen4\font}%
   \def~{\egroup\hbox{\uline{\wd0}\llap{\Tcolor\copy0}}\nobreak\uline\uspace\relax \setbox0=\hbox\bgroup}%
   \leavevmode\coltextA #3 {} }}
\def\coltextA#1 {\ifx^#1^\unskip\else
   \hyphenprocess{#1}\expandafter\coltextB\listwparts\-\end
\expandafter\coltextA\fi}
\def\coltextB#1\-#2\end{\ifx^#2^\coltextC{#1}\else
   \coltextD{#1}\def\next{\coltextB#2\end}\expandafter\next\fi}
\def\coltextC#1{\setbox0=\hbox{#1}\hbox{\uline{\wd0}\hbox{\llap{\Tcolor\copy0}}}\uline\uspace\relax}
\def\coltextD#1{\setbox0=\hbox{#1}\hbox{\uline{\wd0}\llap{\Tcolor\copy0}}\-}

%%%%%%%%%%%%%%%%%%%%%%%%%%%%%%%%%%%%%%%%%%%%%%%%%%%%%%%%%%%%%%%%%%%%%%%%%%%%%%%

\activettchar `

\tit $PAGE_TITLE$

\centerline{\it Petr Olšák, 2013, 2014, 2015}

\bigskip
\centerline{\url{http://petr.olsak.net/opmac-tricks-$LANG_ISO$.html}}

\notoc\nonum
\sec $TOC$

\maketoc

%%%%%%%%%%%%%%%%%%%%%%%%%%%%%%%%%%%%%%%%%%%%%%%%%%%%%%%%%%%%%%%%%%%%%%%%%%%%%%%

\nonum \sec Intro

$INTRO_PAR_1$ OPmac. $INTRO_PAR_2$

$INTRO_PAR_3$


$SECTIONS$

\bye
