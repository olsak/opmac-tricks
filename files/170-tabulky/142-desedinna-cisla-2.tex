anchor-name:dekadic2
id:0171
author:P. O.
date:2019-12-07

----lang:cs--title:Desetinná čísla v tabulkách podruhé
[OPmac trik 0016](#dekadic) pro desetiná čísla v tabulách zakládá „dvojsloupec“, což
působí komplikace např. při použití `\tskip`. Zde je tedy jiné řešení, které
navíc umožňuje mít ve sloupci i nenumerické texty, např. pro záhlaví
tabulky. Řešení zakládá jediný sloupec, tj. nemělo by způsobovat
potíže.

Je vytvořen nový deklarátor `N<číslo>`, např. `N3`, který udává, že ve sloupci
budou čísla (numbers) s desetinnou tečkou, přitom nejvíce cifer za tečkou se
očekává `<číslo>`. Každé číslo se pak doplní o vhodný počet neviditelných
desetinných cifer na stanovený počet a takto jsou čísla zarovnána vpravo.
Tím je dosaženo lícování pod sebou desetinné tečky (která na výstupu může být
čárkou, předefinujete-li `\decimalpoint`). 

Čísla jsou vysázena v matematickém módu, tj. znak „mínus“ vypadá, jak má.

Je-li v položce text bez tečky, je vysázen v textovém módu a je centrován 
(když je menší než celková šířka čísel ve sloupci) nebo je zarovnán vpravo 
(když je větší než celková šířka čísel). 
Je možné přidat další neexistující cifry, tj. `<číslo>` může být
větší než skutečný počet cifer, pokud se tím vizuálně zlepší např. umístění
širšího nenumerického textu. Příklad:

```
\table{c|N3}{
  Ingredients      & w/w (\%) \crl \tskip.2em
  Water            & -97.5    \cr
  Agar powder      &  2.0     \cr
  Solidum chloride &  0.43 
}
```
Je-li cifer za tečkou více než devět, musíte použít složené závorky:
`N{12}`. Také můžete stanovit necelý počet cifer, např. `N{2.5}`, je-li to k
něčemu dobré. Implementace deklarátoru `N` je následující:

```
\def\decimalpoint{.}

\def\paramtabdeclareN#1{\tabiteml\hfil\aligndecdigit#1##.\relax\tabitemr}
\def\aligndecdigit#1#2.#3{%
   \ifx\relax#3\hfil#2\unsskip\hfil \else
   \dimen0=.5em \dimen0=#1\dimen0 $#2$\decimalpoint
   \expandafter\aligndecdigitA\expandafter#3\fi
}
\def\aligndecdigitA#1.\relax{\hbox to\dimen0{$#1$\hss}}

\def\tableA#1#2{\offinterlineskip \colnum=0 \def\tmpa{}\tabdata={}\scantabdata#1\relax
   \def\tmpb{#2}\replacestrings{\crl}{\crcr\crl}%
      \replacestrings{\crli}{\crcr\crli}\replacestrings{\crll}{\crcr\crll}%
      \replacestrings{\crlli}{\crcr\crlli}\replacestrings{\crlp}{\crcr\crlp}%
   \halign\expandafter{\the\tabdata\cr\tmpb\crcr}\egroup}

```

Deklarátor je definován jako `\paramtabdeclareN`. Pomocí `\aligndecdigit` se
přečte text až po tečku a vysází se v matematickém módu a pomocí
`\aligndecdigitA` se vysází zbytek.

Položky s užitím deklarátoru `N` musejí bezpodmínečně končit znakem `&` nebo
`\cr`, nikoli makrem, které symbol konce položky obsahuje. Protože v manuálu
OPmac jsou zmíněna marka `\crl` (apod.), která obsahují `\cr`, je
předefinována přípravná čast tabulky, kde se pomocí `\replacestrings`
nahradí `\crl` za `\crcr\crl`, což udělá nakonec v tabulce stejnou práci. Druhá
možnost je nepoužívat předefinování `\tableA` a ukáznit uživatele, ať před
každé `\crl` (a jim podobné) předřadí `\cr`, pokud má sloupec `N` v tabule
jako poslední.


----lang:en--title:The decimal point aligned, another solution
The [Opmac trick 0016](#dekadic) solves the aligning of decimal
digits in tables with respect to decimal points using two internal columns.
But this solution is incompatible with using of `\tskip`, for example. This
is a reason of this new one-column solution. 

The new declarator of table column `N<number>` is created (for example
`N3`). The `<number>` denotes the maximal number of digits after decimal
point used in printed numbers of declared column. Each printed number si completed to this
maximal `<number>` using invisible digits and such modified numbers are aligned 
to right. So, the decimal points are aligned. The output can be decimal comma
instead decimal point (usable in other languages than English) if you re-define the
`\decimalpoint` macro.

The numbers are printed in math mode, so the minus character is printed as
real minus.

If the printed item does not include any dot then such text is printed in
text mode and it is aligned to center with respect to all decimal numbers in
the column. But if such non-dot text is wider than decimal numbers then it is
right-aligned. You can declare more decimal digits than they are in real, for
example if you need to print wider non-dot text visually better than
simply right-aligned. Example:

```
\table{c|N3}{
  Ingredients      & w/w (\%) \crl \tskip.2em
  Water            & -97.5    \cr
  Agar powder      &  2.0     \cr
  Solidum chloride &  0.43 
}
```
If the number of digits after decimal points are more than 9 then you must
use brackets: `N{12}`. Finally, you can use non-integer number of decimal
digits (`N{2.5}` for example), if you need to do fine tuning of aligning.
The implementation follows:

```
\def\decimalpoint{.}

\def\paramtabdeclareN#1{\tabiteml\hfil\aligndecdigit#1##.\relax\tabitemr}
\def\aligndecdigit#1#2.#3{%
   \ifx\relax#3\hfil#2\unsskip\hfil \else
   \dimen0=.5em \dimen0=#1\dimen0 $#2$\decimalpoint
   \expandafter\aligndecdigitA\expandafter#3\fi
}
\def\aligndecdigitA#1.\relax{\hbox to\dimen0{$#1$\hss}}

\def\tableA#1#2{\offinterlineskip \colnum=0 \def\tmpa{}\tabdata={}\scantabdata#1\relax
   \def\tmpb{#2}\replacestrings{\crl}{\crcr\crl}%
      \replacestrings{\crli}{\crcr\crli}\replacestrings{\crll}{\crcr\crll}%
      \replacestrings{\crlli}{\crcr\crlli}\replacestrings{\crlp}{\crcr\crlp}%
   \halign\expandafter{\the\tabdata\cr\tmpb\crcr}\egroup}

```
The `N` declarator is defined by `\paramtabdeclareN` macro, it means
a declarator with parameter. The `\aligndecdigit` macro prints first part of
decimal number and `\aligndecdigitA` does the rest.

The items must be terminated by `&` or by `\cr` when `N` column is created. 
Using a macro which includes such terminators is unsufficient.
But OPmac manual menitons macros `\crl`, `\crli` etc., they include `\cr`
terminator. So, we re-define the `\Atable` preprocessor in order to each
occurrence of such `\crl` (and others) is replaced by `\crcr\crl` which
does the same work in the table. There is another way: you need not to
redefine `\tableA` macro but you must to type `\cr\crl` instead `\crl`
itself when the `N` column is the last column of the table.
